\section{Introducción}
Los SR aparecen a mediados los 80 de manera natural, por los vendedores que buscaban entender lo que sus clientes deseaban comprar, se generaban catálogos impresos y se entregaban con cupones dentro, en los hogares de los clientes. Con la entrada del internet y el auge del correo electrónico se inició el acercamiento usando listas de correo, muy pronto se vio la necesidad de sumar clientes o visitas de ellos al sitio web corporativo del negocio para dar variedad al catálogo. Gracias a los avances en los algoritmos de minería de datos se observó que algunas herramientas estadísticas podían usarse para anticiparse a las preferencias si conocíamos el perfil de otros usuarios con características similares. Fue así que iniciaron como simple observación de las preferencias de los
clientes y en un inicio sólo  se ofrecían listas de los artículos más comprados por otros usuarios de temporada. Con frecuencia los individuos confiaban en las
recomendaciones ofrecidas por otros; de la misma manera que alguien
te recomienda un libro o una película. La escala de evaluación comenzó
en la mayoría de los productos con una escala fija (A = bueno, B =
regular, C = malo) en donde se establece el valor de cierta variable
(calidad, precio, disponibilidad).
Los sistemas de recomendación hoy en día juegan un rol importante en todos los sitios web. La meta de ellos
es incrementar las ventas, visitas, variedad de contenidos y presentar experiencias de usuario
personalizadas ofreciendo sugerencias para artículos desconocidos
potencialmente interesantes para un usuario.

Los SR actuales utlizan algoritmos avanzados y se invierten grandes recuersos en su desarrollo.

El interés en esta área permanece alto debido a que constituye un
problema rico en investigación y a la abundancia de aplicaciones
prácticas que ayuden a los usuarios a lidiar con sobrecarga de
información.

Las grandes compañías de medios fueron las primeras en invertir en
SR. En 2006 Netflix lanzó una competencia para mejorar su entoces sistema de recomendación en al menos un 10\% de efectividad.

El rol clave de los sistemas de recomendación es que se basan en mineria de datos, rama de la computación que se encuentra en auge y que utiliza algoritmos de inteligencia artificial obteniendo conocimiento a partir de la información.

Sin embargo, a pesar de todos estos avances, la actual generación de
sistemas de recomendación evolucioana rapidamente presentando retos imposibles de resolver por la arquitectura computacional requerida como consecuencia esto a llevado a una competencia entre las empresas por vender la 'mejor solución'. Lo que se busca al diseñar un SR es que se adapte a la inmensa cantidad de información generada por los usuarios optimizando la memoria requerida. Los SR se aplican a un rango amplio de casos como
recomendaciones vacacionales, ciertos tipos de servicios bancarios o de
financiamiento a inversionistas, y productos a ser vendidos en una
tienda creada por un 'carrito inteligente'. Estas mejoras
incluyen mejores métodos para representar comportamiento y la
información acerca de los artículos ha ser adquiridos, métodos avanzados
de recomendación, incorporación de información contextual y utilización
de ratings multicriterio, además del desarrollo de métodos menos
intrusivos que también se apoyan en métricas para determinar desempeño
de los sistemas de recomendación.

\begin{itemize}

\item La recolección de preferencias de los usuarios: \textbf{ No
tiene nada que ver con los perfiles de usuario} ya que esto se realiza a
través de una encuesta que permite conocer las preferencias de los
usuarios, algunos de ellos mencionan las características deseables de un
artículo específico.

\item Análisis: \textbf{ En esta etapa se detectan
patrones en las opciones seleccionadas por los usuarios. - }
\item Generación de opciones:  Los SR se modifican continuamente debido a que el
usuario interacciona con el catalogo de artículos y el SR debe adaptarse
dinámicamente a dichos cambios.

\item Artículos Recomendados: Los
artículos pueden ser en general cualquier bien o servicio requerido por
un usuario específico. No se requiere que el usuario tenga experiencia
previa con el uso del sistema principal. Sin embargo sus selecciones son
tomadas en cuenta para mejorar la precisión de la recomendación próxima.

\end{itemize}

\subsection{Carácteristicas clave que un  SR debería cumplir}

\begin{itemize}
\item
  \textbf{Incrementar el número de artículos vendidos}: Debería ser capaz
  de vender un conjunto de artículos de modo que puedan ser comprados
  sin la intervención de los SR, es decir puede tener su propia meta de
  venta (** ningún visitante se puede ir sin comprar **).
\item
  \textbf{Vender artículos diversos}: Se prefiere la diversidad de
  artículos al ofertar productos ya que las empresas buscan que los
  usuarios (clientes) detecten productos en los que ni siquiera han
  pensado adquirir. Con frecuencia se dan descuentos o rebajas en ellos
  lo que ocasiona que las recomendaciones de los usuarios impacten su
  venta.
\item
  \textbf{Incrementar la satisfacción del usuario}: Un SR bien diseñado
  cambia la interfaz de usuario según las preferencias de los mejores
  clientes, ofreciendo objetivos resaltados y posibilidad de que en base
  a los cambios de la interfase se crean grupos de interés para ofertar
  productos.
\item
  \textbf{Mejor entendimiento de lo que el usuario quiere}: El sondeo
  adecuado de las preferencias del usuario, permite afinar los
  parámetros del SR con el fin de acertar en el ``mejor'' producto.
\item
  \textbf{Incrementar la fidelidad del usuario}: La interacción por
  parte del usuario con el sitio permite que la información sea dinámica
  (contenido que mantenga la atención) con frecuencias las sugerencias y
  reseñas de un producto mantienen al usuario mas tiempo en el sitio lo
  que se aprovecha dando mas opciones de compra.
\end{itemize}

\subsection{Clasificación de los SR}

Los SR usualmente son clasificados en las siguientes categorías:

\begin{itemize}

\item
  \textbf{Recomendaciones Basadas en contenido}: Al usuario le serían
  recomendados artículos similares a los que selecciona en el pasado.
\item
  \textbf{Recomendaciones Colaborativas}: Al usuario le serían
  recomendados artículos que gustan a las personas con preferencias y
  gustos similares en el pasado.
\item
  \textbf{Aproximación Híbrida}: Estos métodos combinan métodos
  colaborativos y basados en contenido.
\end{itemize}

Adicionalmente los sistemas de recomendación que predicen valores
absolutos de rating que usuarios individualmente no han marcado aun en
artículos no conocidos, se les conoce como \emph{filtrado basado en
preferencias }.