\section{Antecedentes}

Las raíces de los sistemas de recomendación inician con trabajos en
ciencia cognitiva, recuperación de información y algunas conexiones con
administración científica, emergen como un área independiente a mediados
de 1990 cuando los investigadores se enfocan en problemas de
recomendación que explícitamente se basaban en una estructura de rating.
Intuitivamente, esta estimación es usualmente basada en la escala
definida por un usuario acerca de una breve información. A partir del
rating de algunos artículos se puede determinar el rating de algunos que
no han sido seleccionados, con el \textbf{rating superior estimado} . De
manera formal el problema de recomendación puede ser formulado como
sigue: Sea $C$ el conjunto de todos los usuarios y sea $S$ el
conjunto de los posibles artículos que pueden ser recomendados tales
como libros, películas o restaurantes. El espacio $S$ de los posibles
artículos puede ser muy amplio, alcanzando los cientos de millones de
artículos. Similarmente el espacio del usuario puede ser bastante
amplio. Sea $u$ la función de utilidad que mide el beneficio de un
articulo $s$ al usuario. De modo que $C \times S \rightarrow R$,
donde $R$ es la totalidad de un conjunto ordenado. Entonces, para cada
usuario $c \in C$, queremos seleccionar tal $s' \in S $ que
maximiza la utilidad del usuario. De manera simplificada tenemos que:
$\forall c \in C, s'=arg max u(c,s)$

En un sistema de recomendación la utilidad de un artículo es usualmente
<<<<<<< HEAD
representada por un \emph{rating} el cual indica como a un usuario le gusta un artículo en particular. Por ejemplo: Juan Perez le dio a
``Harry Potter'' el rating de 7 (en escala de 1 a 10).

\textbf{Ratings}. Rotten Tomatoes (Tomatómetro): El rating del
tomatómetro se basa en las opciones publicadas por críticos de cine y
televisión, es una medida confiable de la calidad de una película y
representa el porcentaje de reseñas positivas dadas a una película.
=======
representada por un \emph{rating} el cual indica como a un usuario
particular le gusta un artículo en particular. Juan Perez le dio a
``Harry Potter'' el rating de 7 (en escala de 1 a 10).

\textbf{Ratings}. Rotten Tomatoes (Tomatómetro): El rating del
tomatometro se basa en las opciones publicadas por críticos de cine y
televisión, es una medida confiable de la calidad de una película y
representa el porcentaje de reseñas positivas dadas a una película,
>>>>>>> origin/master

\textbf{Filtrado Colaborativo}: La idea detrás del filtrado colaborativo
es que se pueden usar los rating de los usuarios que comparten gustos
similares para predecir los que aún no han sido definidos. Para obtener
intuición, se comparan los ratings por pares del usuario

\subsection{ Ejemplos de SR:}

\begin{itemize}

<<<<<<< HEAD
\item Airbnb. Sitio de recomendación de hospedaje. 
Promueve el hospedaje en casas o departamentos de particulares que ofrecen habitaciones a bajo precio donde además de mostrar ubicaciones disponibles por fecha y ubicación preferida, se incluye información de reseñas de clientes previos y se mezclan con comentarios de redes sociales relacionadas al perfil del usuario.

\item Yelp. Recomendación de restaurantes. 
Los usuarios publican reseñas de sitios como: Restaurantes, Tiendas, Servicios (Taxi, Tintoreria, Lavanderia) y van construyendo confianza en los proveedores o vendedores de cierto bien, después se publican en un portal donde se localizan ubicandolos por ubicación cercana al cliente.

\item Los SR de grandes empresas como
Google Play, Apple Movies  utilizan un sistema de ranking para sugerir aplicaciones. Un sistema de ranking puede utilizar elementos como el número de descargas, el tamaño de la aplicación, y la ubicación geográfica del vendedor para mostrar en primer lugar las aplicaciones que tienen la misma categoría y que han sido seleccionadas por otros usuarios en compras previas. De manera similar al ranking de las busquedas de google, se cuentan las palabras clave (keywords) y el numero de links a un sitio específico. De ahi que sólo se muestran los primeros 10 sitios más visitados. 
=======
\item Airbnb. Sitio de recomendación de hospedaje. Promueve el hospedaje en casas o departamentos de particulares que ofrecen habitaciones a bajo precio donde además de mostrar ubicaciones disponibles por fecha y ubicación preferida, se incluye información de reseñas de clientes previos y se mezclan con comentarios de redes sociales relacionadas al perfil del usuario.

\item Yelp. Recomendación de restaurantes. Los usuarios publican reseñas de sitios como: Restaurantes, Tiendas, Servicios (Taxi, Tintoreria, Lavanderia) y van construyendo confianza en los proveedores o vendedores de cierto bien, después se publican en un portal donde se localizan ubicandolos por ubicación cercana al cliente.

\item Los SR de grandes empresas como
Google Play, Apple Movies  utilizan un sistema de ranking para sugerir aplicaciones. Un sistema de ranking puede utilizar elementos como el número de descargas, el tamaño de la aplicación, y la ubicación geográfica del vendedor para mostrar en primer lugar las aplicaciones que tienen la misma categoría y que han sido seleccionadas por otros usuarios en compras previas. De manera similar al ranking de las busquedas de google,  se cuentan las palabras clave (keywords) y el numero de links a un sitio específico.De ahi que sólo se muestran los primeros 10 sitios más visitados. 
>>>>>>> origin/master
\end{itemize}
