\section{Importancia de los SR}

El interés  de las empresas y universidades en las SR se basa en que  constituye un
problema rico en investigación  debido las  aplicaciones
prácticas que ayuden a los usuarios a lidiar con sobrecarga de
información y el beneficio económico que se puede obtener de ellas.

Las grandes compañías de medios fueron las primeras en invertir en
máquinas de aprendizaje comerciales. En 2006 Netflix anunció su máquina
de aprendizaje y la competencia de minería de datos Netflix Prize con 1 millón de
dólares para el equipo que logrará mejorar su entonces recomendador, el cual fue reclamado en 2009, con toda la atención de
los medios, lo que se conoció como `Recomendaciones de Netflix: Más allá
de las 5 estrellas', reveló  que la ciencia de los datos (Science Data) puede ser una grán inversión para cualquier negocio en linea y que la ciencias de la computación en la rama de aprendizaje automático, tiene algoritmos que pueden ser adaptados sistema comercial. La meta de Netflix Prize
fue fondear un algoritmo de recomendaciones que pudiera entregar 10\% de
mejora en precisión de predicción sobre el sistema existente para ello utilizaron herramientas conocidas como Deep Learning. Apple basa
su sistema de recomendaciones de estrenos en el sistema de crítica
Rotten Tomatoes. Google Play Store en un sistema de ranking de
aplicaciones.

El rol clave de los sistemas de recomendación resulta en una vasta
cantidad de investigación en este campo

Sin embargo, a pesar de todos estos avances, la actual generación de
sistemas de recomendación aún requieren mejoras para realizar métodos
más efectivos y aplicables a un rango amplio de casos como
recomendaciones vacacionales, ciertos tipos de servicios bancarios o de
financiamiento a inversionistas, y productos a ser vendidos en una
tienda creada por un ``carrito inteligente''. 
Estas mejoras
incluyen  métodos para representar comportamiento  de los usuarios y la
información acerca de los artículos van ser adquiridos, métodos avanzados
de recomendación, incorporación de información contextual (Ubicación Geográfica, Usuarios satisfechos del producto, etc) y utilización
de ratings multicriterio. Además del desarrollo de métodos no
intrusivos que también se apoyan en métricas para determinar desempeño
de los sistemas de recomendación.
